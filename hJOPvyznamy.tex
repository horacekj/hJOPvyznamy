\documentclass[12pt,a4paper,landscape]{article}

% ===== LOADING PACKAGES =====
% language settings, main documnet language last
\usepackage[czech]{babel}
% enabling new fonts support (nicer)
\usepackage{lmodern}
% setting input encoding
\usepackage[utf8]{inputenc}
% setting output encoding
\usepackage[T1]{fontenc}
% set page margins
\usepackage[top=1cm, bottom=1cm, left=1cm, right=1cm]{geometry}
% package to make bullet list nicer
%\usepackage{enumitem}
% list with smaller distances
%\usepackage{paralist}
\usepackage{enumitem}
% multiple columns on page
\usepackage{multicol}
% change spacing
\usepackage{titlesec}
\begin{document}

\pagestyle{empty}

\begin{center}
\huge Významy povelů hJOP\\
\normalsize Jan Horáček (jan.horacek@kmz-brno.cz)\\
v1.1, \today
\end{center}

% Uprava okraju section a subsection, aby se vse veslo na jednu stranku
\titlespacing*{\section} {0pt}{1.5ex plus 1ex minus .2ex}{0.3ex plus .2ex}
\titlespacing*{\subsection} {0pt}{1.75ex plus 1ex minus .2ex}{0.5ex plus .2ex}

%\title{Významy povelů v~JOP}
%\author{Jan Horáček\\
%	\normalsize jan.horacek@kmz-brno.cz}
%\maketitle

% Vlastni itemize s vlastnimi okraji
\newlist{compactitem}{itemize}{3} % 3 is max-depth
\setlist[compactitem]{label=\textbullet, nosep}
\setlist[compactitem]{label=\textbullet, leftmargin=1em, labelindent=0.1em, itemsep=0em, parsep=0em}

\begin{multicols}{3}

\section{Dopravní kancelář}
\begin{compactitem}
	\item \texttt{MP} -- přepnout stanici na místní provoz
	\item \texttt{DP} -- přepnout stanici na dálkový provoz
	\item \texttt{OSV} -- otevřít menu osvětlení
	\item \texttt{LOKO} -- otevřít menu lokomotiv
	\item \texttt{MSG} -- otevřít okno zpráv
	\item \texttt{CAS>} -- spustit modelový čas
	\item \texttt{CAS<} -- zastavit modelový čas
	\item \texttt{CAS} -- nastavit modelový čas	a~zrychlení
	\item \texttt{NUZ>} -- nouzově uvolnit závěr vybraných bloků
	\item \texttt{NUZ<} -- zrušit nouzové uvolnění závěrů
	\item \texttt{KC} -- rozsvítit přivolávací návěst
	\item \texttt{INFO} -- zobrazit informace o stanici
\end{compactitem}

\subsection{Menu lokomotiv}
\begin{compactitem}
	\item \texttt{NOVÁ loko} -- vytvořit novou lokomotivu
	\item \texttt{EDIT loko} -- editovat lokomotivu
	\item \texttt{SMAZAT loko} -- smazat lokomotivu
	\item \texttt{PŘEDAT loko} -- předat lokomotivu do stanice
	\item \texttt{HLEDAT loko} -- vyhledat lokomotivu
	\item \texttt{RUČ loko} -- lokomotivu do ručního ovladače
\end{compactitem}

\section{Výhybka}
\begin{compactitem}
	\item \texttt{S+} -- přestavit výhybku do polohy plus
	\item \texttt{S-} -- přestavit výhybku do polohy mínus
	\item \texttt{NS+} -- nouzově přestavit výh. do polohy plus
	\item \texttt{NS-} -- nouzově přestavit výh. do polohy mínus
	\item \texttt{STIT} -- editovat štítek výhybky
	\item \texttt{VYL} -- editovat výluku výhybky
	\item \texttt{ZAV>} -- zavést nouzový závěr
	\item \texttt{ZAV<} -- zrušit nouzový závěr
	\\
	\\
\end{compactitem}
	
\section{Úsek}
\begin{compactitem}
	\item \texttt{NOVÝ vlak} -- zavést novou soupravu
	\item \texttt{EDIT vlak} -- editovat soupravu
	\item \texttt{ZRUŠ vlak} -- odstranit soupravu z kolejiště
	\item \texttt{UVOL vlak} -- odstranit soupravu z bloku
	\item \texttt{VEZMI vlak} -- navrátit vlak do kontroly počítače
	\item \texttt{PŘESUŇ vlak>} -- přesunout vlak na jinou kolej
	\item \texttt{PŘESUŇ vlak<} -- ukončit přesouvání vlaku
	\item \texttt{RUČ vlak} -- lokomotivu do ručního ovladače
	\item \texttt{RBP} -- zrušit poruchu blokové podmínky v úseku

	\item \texttt{STIT} -- editovat štítek úseku
	\item \texttt{VYL} -- editovat výluku úseku

	\item \texttt{ZAST>} -- aktivovat zastávku
	\item \texttt{ZAST<} -- deaktivovat zastávku
	\item \texttt{JEĎ vlak} -- odjet s vlakem ze zastávky
	
	\item \texttt{KC} -- konec jízdní cesty
	\item \texttt{VB} -- variantní bod jízdní cesty
	
	\item \texttt{NUZ>} -- nouzově uvolnit závěr
	\item \texttt{NUZ<} -- zrušit nouzové uvolnění závěru	
\end{compactitem}	

\section{Návěstidlo}
\begin{compactitem}
	\item \texttt{VC>} -- zavést počátek vlakové cesty
	\item \texttt{VC<} -- zrušit počátku vlakové cesty
	\item \texttt{PC>} -- zavést počátek posunové cesty
	\item \texttt{PC<} -- zrušit počátek posunové cesty
	\item \texttt{ZAM>} -- zamknout návěstidlo na návěst STŮJ
	\item \texttt{ZAM<} -- odemknou návěstidlo
	\item \texttt{STUJ} -- zrušit návěst
	\item \texttt{DN} -- dodatečná návěst
	\item \texttt{RC} -- zrušit jízdní cestu
	\item \texttt{AB>} -- zavést automatické stavění jízdní cesty
	\item \texttt{AB<} -- zrušit automatické stavění jízdní cesty
	\item \texttt{PN>} -- zapnout přivolávací návěst
	\item \texttt{PN<} -- zrušit přivolávací návěst
	\item \texttt{PP>} -- počátek nouzové posunové cesty
	\item \texttt{PP<} -- zrušení počátku nouzové posunové cesty
	\item \texttt{PPN} -- prodloužit svícení přivolávací návěsti
	\item \texttt{RNZ} -- zrušit nouzové závěry po nouzové cestě
\end{compactitem}	

\section{Přejezd}
\begin{compactitem}
	\item \texttt{UZ} -- uzavřít přejezd
	\item \texttt{ZUZ} -- zrušit uzavření přejezdu
	\item \texttt{NOT>} -- nouzově otevřít přejezd
	\item \texttt{NOT<} -- zrušit nouzové otevření přejezdu
	\item \texttt{STIT} -- editovat štítek přejezdu	
\end{compactitem}

\section{Úvazka}
\begin{compactitem}
	\item \texttt{ZTS>} -- žádost o traťový souhlas
	\item \texttt{ZTS<} -- zrušit žádost o traťový souhlas
	\item \texttt{UTS} -- udělit traťový souhlas
	\item \texttt{ZAK>} -- zavést zákaz odjezdu do trati
	\item \texttt{ZAK<} -- zrušit zákaz odjezdu do trati
	\item \texttt{OTS} -- odmítnout žádost o traťový souhlas
	\item \texttt{STIT} -- editovat štítek úvazky
\end{compactitem}

\section{Zámek}
\begin{compactitem}
	\item \texttt{UK} -- uvolnit klíč
	\item \texttt{ZUK} -- zrušit uvolnění klíče
	\item \texttt{STIT} -- editovat štítek zámku
\end{compactitem}

\section{Zásobník}
\begin{compactitem}
	\item \texttt{PV} -- přímá (přednostní) volba
	\item \texttt{VZ} -- volba do zásobníku
	\item \texttt{EZ} -- editace zásobníku
\end{compactitem}

\end{multicols}

\end{document}